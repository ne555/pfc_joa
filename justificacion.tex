\section*{Motivación}
En el área del reconocimiento de patrones, un tópico de creciente interés
es el reconocimiento de gestos. Esta tarea consiste en la captura del movimiento
corporal de una persona mediante cámaras u otros dispositivos, y la
identificación de poses o movimientos particulares de la cabeza, mano, brazos, etc.
Los gestos reconocidos pueden ser empleados como entrada en el
control de equipamiento, ser traducidos a otra forma de información (p. ej.,
en la traducción de lenguaje de señas a voz). Migrar la interacción humano-computadora
a una manera más natural empleando la capacidad humana de
la gesticulación, permitiría el desarrollo de interfaces más efectivas y amigables
[DFAB03] [MA07].

El análisis y clasificación de gestos de la mano plantea algunas ventajas
respecto al análisis de otras partes del cuerpo. La mano facilita la representación
de un gran número de formas según la combinación de apertura y cierre
de los dedos, con un alto grado de libertad [EBN+07]. Estas características la
convierten en una herramienta de interacción muy efectiva, habiendo sido utilizada
en ambientes virtuales de navegación [Bow02], simulación quirúrgica
[LTCK03], generación de arte visual o música [vHB01, Tar05], entre otros.

