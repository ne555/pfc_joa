\documentclass[portrait,a0paper,fontscale=0.277]{baposter}

\usepackage{enumitem}
\setlist{nolistsep}

\usepackage[spanish]{babel}
\usepackage[utf8]{inputenc}
\usepackage{fontenc}
\usepackage{graphicx}
\usepackage{wrapfig}
%\usepackage{capt-of}
\usepackage{captdef}

\usepackage{tikz}
\usetikzlibrary{calc,automata,positioning,arrows}

\tikzstyle{line} = [draw, -latex]

%%%%%%%%%%%%%%%%%%%%%%%%%%%%%%%%%%%%%%%%%%%%%%%%%%%%%%%%%%%%%%%%%%%%%%%%%%%%%%%%
% Save space in lists. Use this after the opening of the list
%%%%%%%%%%%%%%%%%%%%%%%%%%%%%%%%%%%%%%%%%%%%%%%%%%%%%%%%%%%%%%%%%%%%%%%%%%%%%%%%
\newcommand{\compresslist}{%
\setlength{\topsep}{0pt}%
\setlength{\itemsep}{0pt}%
\setlength{\parskip}{0pt}%
\setlength{\parsep}{0pt}%
}


\newcommand{\Marca}[1]{{\bfseries\textcolor{green}{#1}}}
\newcommand{\MarcaErr}[1]{{\bfseries\textcolor{red}{#1}}}

\begin{document}
\begin{poster}
{ 
	background=plain,
	bgColorOne=white,
	borderColor=orange,
	linewidth=2pt,
	headershade=plain,
	headerColorOne=orange,
	boxshade=plain,
	boxColorOne=white,
	boxColorTwo=orange,
	textborder=roundedsmall,
	headerborder=open,
	headershape=smallrounded,
	headerheight=0.07\textheight,
	headerfont=\large\bf\textsc, %Sans Serif
	eyecatcher=true,
	columns=2
}
{ \includegraphics[height=5.0em]{eji_logo} }
{ \Large\bf\textsc{Justificación, Redacción de Objetivos y Definición de Alcances} }
{ \textsc\small{ \large{Bedrij Walter, Benitez Federico} \\ \footnotesize {wbedrij@gmail.com, federicobenitez2@gmail.com} } }  { \includegraphics[height=5.0em]{fich2} }

	\headerbox{Introducción}{name=intro,column=0,row=0}{
	
	Objetivos:
	\begin{itemize}\compresslist	
	\item
	\item
	\item
	\item
	\end{itemize}

	Alcance: 
	\begin{itemize}\compresslist	
	\item 
	\item 
	\item 
	\end{itemize}
	}

\headerbox{Desarrollo }{name=ana,column=0,span=1,below=intro}{

	\section*{Justificación}
	La Justificación permite tomar una decisión sobre si  el proyecto
debe avanzar a la siguiente etapa de desarrollo. 

Evaluar la viabilidad de una propuesta de proyecto.
implica:

\begin{itemize}\compresslist	
	\item Explicar las razones para el proyecto y los beneficios esperados.
	\item Describir el contexto relevante en términos de política, los negocios,
factores económicos y/o de programas.
	\item Describir las interdependencias con otros proyectos en caso de ser necesario.
	\item Indicar un aprendizaje previo relacionado con un aspecto de la justificación del proyecto, y explicar cómo ese aprendizaje podría ser beneficioso para el proyecto.
	\item Dar una evaluación de las medidas de éxito, basado en
criterios reconocidos.
	\end{itemize}

\section*{Objetivos: SMART}
\begin{description}
	\item[Specific (Específico)] Dejar explícito, para que no haya ambigüedad.
		\begin{itemize}\compresslist
			\item ¿Qué hacer? Acciones positivas.
			\item ¿Cuál será el resultado?
			\item ¿Por qué es importante?
			\item ¿Quién es el responsable?
			\item ¿Cuáles son los requerimientos y las restricciones?
		\end{itemize}
	\item[Measurable (Medible)]
	Definir criterios que indiquen que se logró el resultado esperado (o qué tan lejos estamos)
\item[Achievable (Alcanzable)]\hfill
	\begin{itemize}\compresslist
		\item ¿Se puede lograr con los recursos que disponemos?
		\item ¿Entendemos las restricciones?
		\item ¿Es posible y práctico?
	\end{itemize}
\item[Relevant (Relevante)]
	Cómo el objetivo se relaciona con el rol de cada miembro del equipo, y los objetivos de la organización.
	Los objetivos deben apoyarse mutuamente y no generar conflicto.
\item[Time-bound (Limitado en el tiempo)]
	Definir fechas límite.
	%¿Cuándo se completará?

\end{description}
	
	
\section*{Definir el Alcance} Es el proceso que consiste en desarrollar una descripción detallada del
proyecto y del producto. Se elabora a partir de los entregables principales, los supuestos y
las restricciones que se documentan durante el inicio del proyecto. Se analizan los riesgos, los
supuestos y las restricciones existentes, para verificar que estén completos; según sea necesario,
se irán agregando nuevos riesgos, supuestos y restricciones. 

	\begin{center}	
		\includegraphics[width=.8\linewidth]{alcance}
	\end{center}
}

	\headerbox{Ejemplos}{name=ejemplo,column=1,span=1}{
	
	\section*{Justificación}
En el área del reconocimiento de patrones, un tópico de creciente interés
es el reconocimiento de gestos. (...)
 Migrar la interacción humano-computadora
a una manera más natural (...), permitiría el desarrollo de interfaces más efectivas y amigables.

(...). La mano facilita la representación
de un gran número de formas según la combinación de apertura y cierre
de los dedos, (...).

   
    Una vez introducida la tarea de interés y el marco general de trabajo, se
enuncian a continuación los objetivos generales y particulares del presente
Proyecto Final.

El objetivo general de este trabajo es el siguiente:

\begin{itemize}
\item Adquirir nuevos conocimientos acerca de aplicaciones específicas innovadoras
de técnicas de procesamiento digital de imágenes.

\item Aplicar los conocimientos adquiridos en el transcurso de la carrera a
un proyecto que utilice técnicas de procesamiento digital de imágenes,
que realice un aporte ingenieril a la comunidad en general.
\end{itemize}

Los objetivos específicos son los siguientes:
\begin{itemize}
	\item Implementar rutinas de manejo de cámara web para adquisición de flujo de video.
	\item Diseñar e implementar rutinas básicas para definición de la silueta de la mano.
	\item Analizar, seleccionar, implementar y adaptar algoritmos de segmentación, morfología, procesamiento en color y otros para extracción de características de la mano.
	\item Definir e implementar técnicas de reconocimiento de patrones para identificación de la seña.
	\item Diseñar e implementar una interfaz gráfica que haga accesible al usuario final su utilización.
\end{itemize}


	
	\section* {Alcance} 
En este trabajo se presenta el diseño e implementación de un \Marca{sistema para
el reconocimiento} de signos manuales en \Marca{tiempo real}, mediante el procesamiento del flujo de video de una \Marca{cámara web} estándar en un \Marca{ambiente de
trabajo controlado}.

	
	}


  \headerbox{Conclusión}{name=questions,column=1,below=ejemplo}{
		\begin{itemize}\compresslist
			\item
		\end{itemize}
  }



  \headerbox{Referencias}{name=referencias,column=1,above=bottom}{
%%%%%%%%%%%%%%%%%%%%%%%%%%%%%%%%%%%%%%%%%%%%%%%%%%%%%%%%%%%%%%%%%%%%%%%%%%%%%%
    \bibliographystyle{ieee}
    \renewcommand{\section}[2]{\vskip 0.05em}
      \begin{thebibliography}{1}\itemsep=-0.01em
      \setlength{\baselineskip}{0.4em}

	  \bibitem{2004:GPM}
		\newblock{A Guide To The Project Management Body Of Knowledge (PMBOK Guides).}
		\newblock{Project Management Institute.}
		\newblock{2004.}
		
		\bibitem{2007:PMPJP}
		\newblock{Project Management:
Project Justification and
Planning.
}
		\newblock{Scottish Qualifications Authority.}
		\newblock{2007.}	
		
		 \bibitem{2012:FTG}
		 Graham Yemm.
		\newblock{FT Essential Guide to Leading Your Team: How to Set Goals, Measure Performance and Reward Talent.}
		\newblock{Pearson Education Limited.}
		\newblock{2012.}	
		
      \end{thebibliography}
   %\vspace{0.3em}
  }

\end{poster}
\end{document}
