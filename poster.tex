\documentclass[portrait,a0paper,fontscale=0.277]{baposter}

\usepackage{enumitem}
\setlist{nolistsep}

\usepackage[spanish]{babel}
\usepackage[utf8]{inputenc}
\usepackage{fontenc}
\usepackage{graphicx}
\usepackage{wrapfig}
%\usepackage{capt-of}
\usepackage{captdef}

\usepackage{tikz}
\usetikzlibrary{calc,automata,positioning,arrows}

\tikzstyle{line} = [draw, -latex]

%%%%%%%%%%%%%%%%%%%%%%%%%%%%%%%%%%%%%%%%%%%%%%%%%%%%%%%%%%%%%%%%%%%%%%%%%%%%%%%%
% Save space in lists. Use this after the opening of the list
%%%%%%%%%%%%%%%%%%%%%%%%%%%%%%%%%%%%%%%%%%%%%%%%%%%%%%%%%%%%%%%%%%%%%%%%%%%%%%%%
\newcommand{\compresslist}{%
\setlength{\topsep}{0pt}%
\setlength{\itemsep}{0pt}%
\setlength{\parskip}{0pt}%
\setlength{\parsep}{0pt}%
}

\renewcommand{\includegraphics}[2][foo]{}

\begin{document}
\begin{poster}
{ 
	background=plain,
	bgColorOne=white,
	borderColor=orange,
	linewidth=2pt,
	headershade=plain,
	headerColorOne=orange,
	boxshade=plain,
	boxColorOne=white,
	boxColorTwo=orange,
	textborder=roundedsmall,
	headerborder=open,
	headershape=smallrounded,
	headerheight=0.07\textheight,
	headerfont=\large\bf\textsc, %Sans Serif
	eyecatcher=true
}
{ \includegraphics[height=5.0em]{eji_logo} }
{ \Large\bf\textsc{Justificación, Redacción de Objetivos y Definición de Alcances} }
{ \textsc\small{ \large{Bedrij Walter, Benitez Federico} \\ \footnotesize {wbedrij@gmail.com, federicobenitez2@gmail.com} } }  { \includegraphics[height=5.0em]{fich2} }

	\headerbox{Introducción}{name=intro,column=0,row=0}{
	
	Objetivos:
	\begin{itemize}\compresslist	
	\item
	\item
	\item
	\item
	\end{itemize}

	Alcance: 
	\begin{itemize}\compresslist	
	\item 
	\item 
	\item 
	\end{itemize}
	}

\headerbox{Desarrollo }{name=ana,column=1,span=2,row=0}{

\begin{description}
\item[La Justificacion] permite tomar una decisión sobre si  el proyecto
debe avanzar a la siguiente etapa de desarrollo 

Evaluar la viabilidad de una propuesta de proyecto.
implica:

\begin{itemize}\compresslist	
	\item Explicar las razones para el proyecto y los beneficios esperados
	\item Describir el contexto relevante en términos de política, los negocios,
factores económicos y / o de programas
	\item Describir las interdependencias con otros proyectos en caso de ser necesario
	\item Indicar un aprendizaje previo relacionado con un aspecto de la justificación del proyecto, y explicar cómo ese apredizaje podría ser beneficioso para el proyecto
	\item dar una evaluación de las medidas de éxito, basado en
criterios reconocidos.
	\end{itemize}

\section*{Objetivos: «SMARTER»}
\begin{description}
	\item[Specific (Específico)] dejar explícito, para que no haya ambigüedad.
		\begin{itemize}\compresslist
			\item ¿Qué hacer? Acciones positivas.
			\item ¿Cuál será el resultado?
			\item ¿Por qué es importante?
			\item ¿Quién es el responsable?
			\item ¿Cuáles son los requerimientos y las restricciones?
		\end{itemize}
	\item[Measurable (Medible)]
	Definir criterios que indiquen que se logró el resultado esperado (o qué tan lejos estamos)
\item[Achievable (Alcanzable)]
	\begin{itemize}\compresslist
		\item ¿Se puede lograr con los recursos que disponemos?
		\item ¿Entendemos las restricciones?
		\item ¿Es posible y práctico?
	\end{itemize}
\item[Relevant (Relevante)]
	Cómo el objetivo se relaciona con el rol de cada miembro del equipo, y los objetivos de la organización.
	Los objetivos deben apoyarse mutuamente y no generar conflicto.
\item[Time-bound (Limitado en el tiempo)]
	Definir fechas límite.
	%¿Cuándo se completará?
%revisar
\item[Evaluated (Evaluado)]
	Debe seguirse el progreso del objetivo.
	%Poder corregir.
\item[Rewarded (Recompensable)]
	Se vuelve un proceso de aprendizaje.
\end{description}
	
	
\item [Definir el Alcance] es el proceso que consiste en desarrollar una descripción detallada del
proyecto y del producto. Se elabora a partir de los entregables principales, los supuestos y
las restricciones que se documentan durante el inicio del proyecto. Se analizan los riesgos, los
supuestos y las restricciones existentes, para verificar que estén completos; según sea necesario,
se irán agregando nuevos riesgos, supuestos y restricciones. 

\end{description}

	\begin{center}	
	\begin{tabular}{ccc}
		\includegraphics[width=.28\textwidth]{fondo.png} &
		\includegraphics[width=.28\textwidth]{hue.png} &
		\includegraphics[width=.28\textwidth]{fondoandhue.png} \\

		Fondo & Hue & Fondo AND Hue \\
	\end{tabular}
	\end{center}
	\begin{center}
	\begin{tabular}{cc}
		\includegraphics[width=.28\textwidth]{final.png} &
		\includegraphics[width=.28\textwidth]{interfaz.png} \\

		Objeto Ganador & Interfaz \\
	\end{tabular}
	\end{center}
}
	\headerbox{Ejemplos}{name=nn,column=1,span=2,below=ana}{

	\begin{center}
	\begin{tabular}{c}
		\begin{tikzpicture}[
	%scale=0.6,
	every node/.style={transform shape},
	node distance=10mm and 5mm,
	nonterminal/.style={
		rectangle,
		rounded corners=3mm,
		minimum size=6mm,
		very thick,
		draw=black,
	},
	terminal/.style={
		rectangle,
		minimum size=2em,
		rounded corners=1mm,
		very thin,
		draw=black,
		font=\footnotesize,
		%text width=15ex,
		align=center,
		%text centered
	},
	point/.style={
		circle,
		inner sep=0pt,
	}
]
	\node (start) [terminal]
		{Fotograma \\de vídeo};
	\node (p0) [point,minimum size=2pt,fill=black,right=of start] {};
	\node (p1) [point,above=of p0] {};
	\node (p2) [point,below=of p0] {};
	\node (skin) [terminal,right=of p1]
		{Enmascaramiento \\de la piel};
	\node (p3) [point,right=of skin] {};
	\node (diff) [terminal,right=of p2]
		{Diferenciación \\con el fondo};
	\node (ref) [terminal,left=of p2]
		{Imagen \\de referencia};
	\node (p4) [point,right=of diff] {};
	%\node (p5) [point,above=of p4] {};
	\node (and) [nonterminal] at ($(p3) !.5! (p4)$)
		{$\times$};
	\node (growth) [terminal,right=of and]
		{Crecimiento inverso \\de regiones};
	\node (face) [terminal,right=of growth]
		{Eliminación \\de la cara};
	\node (pointer) [terminal,right=of face]
		{Ubicación \\del puntero};
	\node (interfaz) [terminal,right=of pointer]
		{Interfaz};

	\draw  (start) -- (p0);
	\draw [line] (p0) |- (skin);
	\draw [line] (p0) |- (diff);
	\draw [line] ([yshift=-1ex]ref.east) -- ([yshift=-1ex]diff.west);
	\draw [line] (skin) -| (and);
	\draw [line] (diff) -| (and);
	\draw [line] (and) -- (growth);
	\draw [line] (growth) -- (face);
	\draw [line] (face) -- (pointer);
	\draw [line] (pointer) -- (interfaz);


\end{tikzpicture}
 \\
		%\includegraphics[width=\textwidth-2em]{diagrama.pdf} \\
		Diagrama en bloques
	\end{tabular}
	\end{center}

	\begin{description}
		\item[Diferenciación con el fondo] \hfill
		\begin{itemize}\compresslist
			\item 
		\end{itemize}
	\end{description}
	}


  \headerbox{Conclusión}{name=questions,below=intro}{
		\begin{itemize}\compresslist
			\item
		\end{itemize}
  }



  \headerbox{Referencias}{name=referencias,column=0,above=bottom}{
%%%%%%%%%%%%%%%%%%%%%%%%%%%%%%%%%%%%%%%%%%%%%%%%%%%%%%%%%%%%%%%%%%%%%%%%%%%%%%
    \bibliographystyle{ieee}
    \renewcommand{\section}[2]{\vskip 0.05em}
      \begin{thebibliography}{1}\itemsep=-0.01em
      \setlength{\baselineskip}{0.4em}

	  \bibitem{2004:GPM}
		\newblock{A Guide To The Project Management Body Of Knowledge (PMBOK Guides).}
		\newblock{Project Management Institute.}
		\newblock{2004.}
		
		\bibitem{2007:PMPJP}
		\newblock{Project Management:
Project Justification and
Planning.
}
		\newblock{Scottish Qualifications Authority.}
		\newblock{2007.}	
		
		 \bibitem{2012:FTG}
		 Graham Yemm.
		\newblock{FT Essential Guide to Leading Your Team: How to Set Goals, Measure Performance and Reward Talent.}
		\newblock{Pearson Education Limited.}
		\newblock{2012.}	
		
      \end{thebibliography}
   %\vspace{0.3em}
  }

\end{poster}
\end{document}
