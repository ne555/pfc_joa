\documentclass[portrait,a0paper,fontscale=0.277]{baposter}

\usepackage[spanish]{babel}
\usepackage[utf8]{inputenc}
\usepackage{graphicx}

%%%%%%%%%%%%%%%%%%%%%%%%%%%%%%%%%%%%%%%%%%%%%%%%%%%%%%%%%%%%%%%%%%%%%%%%%%%%%%%%
% Save space in lists. Use this after the opening of the list
%%%%%%%%%%%%%%%%%%%%%%%%%%%%%%%%%%%%%%%%%%%%%%%%%%%%%%%%%%%%%%%%%%%%%%%%%%%%%%%%
\newcommand{\compresslist}{%
\setlength{\itemsep}{1pt}%
\setlength{\parskip}{0pt}%
\setlength{\parsep}{0pt}%
}

\begin{document}
\begin{poster}
{ 
	background=plain,
	bgColorOne=white,
	borderColor=green,
	linewidth=2pt,
	headershade=plain,
	headerColorOne=green,
	boxshade=plain,
	boxColorOne=white,
	boxColorTwo=green,
	textborder=roundedsmall,
	headerborder=open,
	headershape=smallrounded,
	headerheight=0.1\textheight,
	headerfont=\Large\bf\textsc, %Sans Serif
	eyecatcher=true
}
%{\includegraphics[height=5em]{images/graph_occluded.pdf}} 
{}
{ \bf\textsc{Localización de Manos por webcam \\ aplicado a Interfaz para Dibujo} }
{ \textsc{Bedrij W. Benitez F. Benitez F.} }
{ \includegraphics[height=9.0em]{images/logo} }

	\headerbox{Introducción}{name=intro,column=0,row=0}{
El objetivo del trabajo es desarrollar un algoritmo que permita localizar la posición de las manos y describir su 
trayectoria teniendo como entrada un video obtenido de una cámara web. Siendo luego aplicado a la creación de 
interfaces aéreas sin la utilización de marcadores. El trabajo no aborda el seguimiento de la mano, simplemente se trata 
a cada fotograma de manera independiente. 

No supone mayor penalidad en el tiempo de ejecución, sin embargo imposibilita la resolución de casos en donde la 
mano se encuentra sobre la cara. 

Además se supone que el brazo se encuentra cubierto. Si bien la detección sería mucho más simple (al menos en el caso 
de que no exista superposición), se es difícil la elección de un punto característico 
	}

	\headerbox{Métodos}{name=metodos,column=0,below=intro}{
	\begin{enumerate}\compresslist
		\item Diferenciación con el fondo
		\item Enmascaramiento de la piel
		\item Eliminación de la cara
		\item Ubicación del puntero
	\end{enumerate}
	}

	\headerbox{Resultados}{name=resultadoss,column=1,span=2,row=0}{
		Aca irían las imágenes de los procesos
	}

  \headerbox{Referencias}{name=referencias,column=0,above=bottom}{
%%%%%%%%%%%%%%%%%%%%%%%%%%%%%%%%%%%%%%%%%%%%%%%%%%%%%%%%%%%%%%%%%%%%%%%%%%%%%%
    \bibliographystyle{ieee}
    \renewcommand{\section}[2]{\vskip 0.05em}
      \begin{thebibliography}{1}\itemsep=-0.01em
      \setlength{\baselineskip}{0.4em}
	  \bibitem{gonzales2002:manual}
	  	R. Gonzales, R. Woods
		\newblock{Digital Image Processing}
		\newblock In {\emph Prentice Hall 2002}
      \bibitem{amberg11:graphtrack}
        B.~Amberg, T. Vetter.
        \newblock {GraphTrack}: {F}ast and {G}lobally {O}ptimal {T}racking in {V}ideos
        \newblock In {\em CVPR '11}
      \bibitem{awf:tracking}
        A.~Buchanan and A.~Fitzgibbon.
        \newblock {I}nteractive {F}eature {T}racking using {K-D} {T}rees and {D}ynamic {P}rogramming.
        \newblock In {\em CVPR '06}
      \end{thebibliography}
   \vspace{0.3em}
  }

%@BOOK{Gonzales2002,
%    Address        = {},
%    Author         = {Rafael C. Gonzalez and Richard E. Woods},
%    Publisher      = {Prentice Hall},
%    Title          = {Digital Image Processing},
%    Year           = {2002}
%}
%
%@ARTICLE{Cheng,
%    Author         = {H. D. Cheng and X. H. Jiang and Y. Sun and Jing Li Wang},
%    Journal          = {Pattern Recognition},
%    Pages        = {2259--2281},
%    Title          = {Color Image Segmentation: Advances \& Prospects},
%    Volume         = {34},
%    Year           = {2001}
%}
%
%@ARTICLE{Vogt,
%    Author         = {R. J. Godoy and P. Novara and J. P. Vogt},
%    Journal        = {37 JAIIO},
%    Pages          = {},
%    Title          = {Reconocimiento de señas},
%    Volume         = {},
%    Year           = {2008}
%}
%

  \headerbox{Source Code}{name=source,column=2,above=bottom}{
  \noindent
  \begin{minipage}{\linewidth}
  \begin{minipage}{0.7\linewidth}
    \indent{}The source code and compiled executables with an interactive interface are available at 
	\end{minipage}
	\end{minipage}
  \emph{http://www.cs.unibas.ch/personen/amberg\_brian/graphtrack}
  }

  \headerbox{A Future Direction}{name=questions,column=1,span=1,above=bottom}{
    We incorporated a background model, where a click informs us not only that `this is how the
    patch looks like', but also for the rest of the frame, `this is how the patch
    does not look like'. 
    
    Can we also \emph{efficiently} use a background tracks model, allowing us
    to reason, `this would be a good track, but part of it can be better
    explained by tracking another point'.
   \vspace{0.3em}
  }

\end{poster}



\end{document}
