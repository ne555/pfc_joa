\documentclass[portrait,a0paper,fontscale=0.277]{baposter}
%useless change

\usepackage[spanish]{babel}
\usepackage[utf8]{inputenc}
\usepackage{graphicx}

%%%%%%%%%%%%%%%%%%%%%%%%%%%%%%%%%%%%%%%%%%%%%%%%%%%%%%%%%%%%%%%%%%%%%%%%%%%%%%%%
% Save space in lists. Use this after the opening of the list
%%%%%%%%%%%%%%%%%%%%%%%%%%%%%%%%%%%%%%%%%%%%%%%%%%%%%%%%%%%%%%%%%%%%%%%%%%%%%%%%
\newcommand{\compresslist}{%
\setlength{\itemsep}{1pt}%
\setlength{\parskip}{0pt}%
\setlength{\parsep}{0pt}%
}

\begin{document}
\begin{poster}
{ 
	background=plain,
	bgColorOne=white,
	borderColor=green,
	linewidth=2pt,
	headershade=plain,
	headerColorOne=green,
	boxshade=plain,
	boxColorOne=white,
	boxColorTwo=green,
	textborder=roundedsmall,
	headerborder=open,
	headershape=smallrounded,
	headerheight=0.1\textheight,
	headerfont=\Large\bf\textsc, %Sans Serif
	eyecatcher=true
}
%{\includegraphics[height=5em]{images/graph_occluded.pdf}} 
{}
{ \bf\textsc{Localización de Manos por webcam \\ aplicado a Interfaz para Dibujo} }
{ \textsc{\\ Bedrij W. Benitez F. Benitez F.} }
{ \includegraphics[height=9.0em]{images/logo} }

	\headerbox{Introducción}{name=intro,column=0,row=0}{
	El objetivo del trabajo desarrollar un algoritmo que permita localizar la posición de las manos y describir su trayectoria teniendo como entrada un vídeo sacado de una cámara web.
	Siendo luego aplicado a la creación de interfaces aéreas sin la utilización de marcadores.

	El trabajo no aborda el seguimiento de la mano, simplemente se trata a cada fotograma de manera independiente. 
No supone mayor penalidad en  el tiempo de ejecución, 
sin embargo imposibilita la resolución de casos en donde
la mano se encuentra sobre la cara.

Además se supone que el brazo se encuentra cubierto. 
Si bien la detección sería mucho más simple (al menos en el caso de que no exista superposición), se dificulta la elección de un punto característico.
	}


	\headerbox{Resultados}{name=resultados,column=1,span=2,row=0}{
		El uso de una imagen de referencia para el fondo permitió  que  la  mano  se
		ubicara sobre zonas de tonalidad parecida.

		El  Hue  mostró  ser  bastante  robusto  a  los  cambios  de  intensidad  de
		iluminación, por lo que pudo fijarse un umbral en forma experimental y no  se
		necesitaba de una posterior calibración.

		En caso de que el  dedo  se  considerara  como  una  región  separada  seria
		eliminado por su  área  insignificante,  quedando  solo  la  palma  de  forma
		aproximadamente circular, por lo que la heurística fallaba.

		 Con respecto  a  la  interfaz  observamos  unas  oscilaciones  instantáneas
		producto de detecciones de  puntos  lejanos  no  deseados  con  respecto  al
		centroide de las manos como en el caso cuando se usa  la  interfaz  con  una
		remera corta, como este método se basa en diferencias de imágenes y Hue,  el
		brazo  sobrevive  a  los  procesamientos   de   eliminación,   y   como   es
		aproximadamente del mismo color de la mano,  es  segmentado  obteniendo  así
		puntos lejanos no deseados, por  ello  una  de  las  restricciones  es  usar
		atuendo que oculte el brazo y descubra las manos.
	}

	\headerbox{Métodos}{name=metodos,column=0,span=1,below=intro}{
		\begin{enumerate}\compresslist
			\item Obtención de una imagen de referencia
			\item Captura de los fotogramas del video
			\item Diferenciación con el fondo
			\begin{itemize}
				\item Enmascaramiento de la piel
				\item Crecimiento inverso de regiones
			\end{itemize}
			\item Eliminación de la cara
			\item Ubicación del puntero
			\item Diseño de interfaz
		\end{enumerate}
	}

	\headerbox{Partes que merecen ser explicadas}{name=nn,column=1,span=2,below=resultados}{
		\section{Enmascaramiento de la piel}

			Se decidió ocupar  la información del  Hue,  como  sugiere  [2].  Como  pre-
			proceso se aplica un filtro de mediana,  tratando  además  de  uniformar  la
			textura de la piel. Se aplica un umbral global en la  componente  H  dejando
			pasar el rango $H = 7 \pm 5$ (valores experimentales de la piel).

			Para eliminar los huecos internos en el Hue, se procedió aplicar  un  cierre
			morfológico con un elemento estructurante de forma cuadrada y tamaño de  5x5
			,como se explica en [1].

			Se combina con la máscara de diferencia, obteniéndose  entonces  objetos  no
			pertenecientes al fondo con un tono parecido al de la piel.



		\section{Crecimiento inverso de regiones}

			 Otro método  para  eliminar  los  huecos  dentro  de  las  regiones  es  el
			crecimiento inverso, como se observa en [3].

			Trabajando sobre la máscara binaria del Hue, a partir de  un  punto  que  se
			conoce fuera de cualquier zona de  interés,  realizamos  el  crecimiento  de
			regiones. Con esto obtenemos el fondo, al invertir  la  máscara  se  tendrán
			las siluetas de los objetos.

		\section{Eliminación de la cara}

			Luego del proceso anterior, se tiene una imagen compuesta por  objetos  tipo
			piel. Estas pueden ser la cara, las manos y algún ruido que haya pasado  los
			filtros. Por eso se etiquetan las  regiones, quedándose con las 3  de  mayor
			área, que se supone perteneciente a ambas manos y la cara.  Por  observación
			debieran de tener un área similar, si resulta que  alguna  es  excesivamente
			pequeña con respecto a  las  otras  ($25 \%$)  es  considerada  como  ruido  y
			eliminada.

			Queda discriminar las correspondientes a las manos, es decir, eliminar la
			región que identifica a la cara. Para facilitar el proceso, la mano debe
			tener un dedo extendido. Suponiendo a la cara como redonda, es de esperar
			que al hacer el factor de  redondez
			\[ C = \frac{\makebox{Área}}{\pi r^2}\]
			sea cercano a 1, quedando así la cara diferenciada de las manos. Donde r  es
			el radio del círculo cuyo centro se corresponde al centroide de la región  y
			pasa por su punto más alejado.

	}


  \headerbox{Referencias}{name=referencias,column=0,above=bottom}{
%%%%%%%%%%%%%%%%%%%%%%%%%%%%%%%%%%%%%%%%%%%%%%%%%%%%%%%%%%%%%%%%%%%%%%%%%%%%%%
    \bibliographystyle{ieee}
    \renewcommand{\section}[2]{\vskip 0.05em}
      \begin{thebibliography}{1}\itemsep=-0.01em
      \setlength{\baselineskip}{0.4em}
	  \bibitem{gonzales2002:manual}
	  	R. Gonzales, R. Woods
		\newblock{Digital Image Processing}
		\newblock In {\emph Prentice Hall 2002}
      \bibitem{amberg11:graphtrack}
        B.~Amberg, T. Vetter.
        \newblock {GraphTrack}: {F}ast and {G}lobally {O}ptimal {T}racking in {V}ideos
        \newblock In {\em CVPR '11}
      \bibitem{awf:tracking}
        A.~Buchanan and A.~Fitzgibbon.
        \newblock {I}nteractive {F}eature {T}racking using {K-D} {T}rees and {D}ynamic {P}rogramming.
        \newblock In {\em CVPR '06}
      \end{thebibliography}
   \vspace{0.3em}
  }

%@BOOK{Gonzales2002,
%    Address        = {},
%    Author         = {Rafael C. Gonzalez and Richard E. Woods},
%    Publisher      = {Prentice Hall},
%    Title          = {Digital Image Processing},
%    Year           = {2002}
%}
%
%@ARTICLE{Cheng,
%    Author         = {H. D. Cheng and X. H. Jiang and Y. Sun and Jing Li Wang},
%    Journal          = {Pattern Recognition},
%    Pages        = {2259--2281},
%    Title          = {Color Image Segmentation: Advances \& Prospects},
%    Volume         = {34},
%    Year           = {2001}
%}
%
%@ARTICLE{Vogt,
%    Author         = {R. J. Godoy and P. Novara and J. P. Vogt},
%    Journal        = {37 JAIIO},
%    Pages          = {},
%    Title          = {Reconocimiento de señas},
%    Volume         = {},
%    Year           = {2008}
%}
%


  \headerbox{A Future Direction}{name=questions,below=metodos,above=referencias}{
		La detección de la posición de las manos fue posible en tiempo real,
		si bien se usaron suposiciones como que los brazos estuvieran cubiertos
		y de que no ocurriera superposición.

		Los criterios de discriminación de las manos obligó a 
		usar una conformación en las posiciones de los dedos. 

		Problemas en la segmentación generaba falsas identificaciones, 
		lo que se traducía en discontinuidades en las posiciones de los punteros.

		Para intentar solventar esos problemas, se propone realizar tracking
		de las manos, requiriendo simplemente una detección global inicial.

		Se puede realizar un refinamiento en la selección del punto característico
		correspondiente a las manos. Por ejemplo, poder discriminar los dedos
  }

\end{poster}



\end{document}
