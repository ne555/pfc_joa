\section*{Objetivos: «SMARTER»}
\begin{description}
	\item[Specific (Específico)] dejar explícito, para que no haya ambigüedad.
		\begin{itemize}\compresslist
			\item ¿Qué hacer? Acciones positivas.
			\item ¿Cuál será el resultado?
			\item ¿Por qué es importante?
			\item ¿Quién es el responsable?
			\item ¿Cuáles son los requerimientos y las restricciones?
		\end{itemize}
	\item[Measurable (Medible)]
	Definir criterios que indiquen que se logró el resultado esperado (o qué tan lejos estamos)
\item[Achievable (Alcanzable)]
	\begin{itemize}\compresslist
		\item ¿Se puede lograr con los recursos que disponemos?
		\item ¿Entendemos las restricciones?
		\item ¿Es posible y práctico?
	\end{itemize}
\item[Relevant (Relevante)]
	Cómo el objetivo se relaciona con el rol de cada miembro del equipo, y los objetivos de la organización.
	Los objetivos deben apoyarse mutuamente y no generar conflicto.
\item[Time-bound (Limitado en el tiempo)]
	Definir fechas límite.
	%¿Cuándo se completará?
%revisar
\item[Evaluated (Evaluado)]
	Debe seguirse el progreso del objetivo.
	%Poder corregir.
\item[Rewarded (Recompensable)]
	Se vuelve un proceso de aprendizaje.
\end{description}
